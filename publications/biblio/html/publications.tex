\documentclass[a4paper]{article}
\usepackage[latin1]{inputenc}
\usepackage[T1]{fontenc}
\usepackage[francais,english]{babel}
\usepackage{times,bibunits,wasysym,amstext}

%\def\bibcite#1#2{}
\usepackage[dvips]{hyperref}
%\newcommand\htmladdnormallink[2]{\href{#1}{#2}}
\usepackage{perso2e,LienPDF}

\let\ifFTR=\iftrue
%\newcommand\fup[1]{$^{\text{#1}}$}

{\catcode`\~=12\catcode`\:=12\catcode`\_=12\catcode`\&=12
  \gdef\RacineHTML{http://ronan.keryell.fr}%
}
\makeatletter
\newcommand{\FDP@space}{\space}
\makeindex
\def\indexbib#1{}

%% R�cup�re le num�ro de version RCS :
\def\getVersion Revision: #1 {\def\Version{#1}}
\def\getDollars$#1${\getVersion#1}
\getDollars$Revision: 1.33 $

\select@language{english}


\begin{document}

\title{Publications version \Version}

\author{\href{\RacineHTML}{Ronan Keryell}}

\maketitle

\begin{itemize}
\item Version PostScript :
  \LienPDF{\RacineHTML/publications/biblio/publications.ps}
\item Version PDF :
  \LienPDF{\RacineHTML/publications/biblio/publications.pdf}
\item Version HTML :
  \LienPDF{\RacineHTML/publications/biblio/html}
\end{itemize}

\tableofcontents



\section{Compilation \& Parallelization for HPC and Embedded Systems ---
  \emph{Compilation \& parall�lisation}}
\label{sec:comp-et-parall}

\begin{bibunit}[liste_publications]
  \nocite{OpenGPU2014:coding_rules,SMECY:book2013:2,TSI2013:terapix,TSI2012:compil_static_host_accel,TSI2012:polyhedre_compilation,EW2012:SMECY,Rapido2012:SESAM_Par4All,IMPACT2012:regions_to_heterogeneous_computing,LCPC2011:comm_optim,RenPar2011:poly_compil,RenPar2011:optim_comm,RenPar2011:compil_heter,IMPACT2011:PIPS_also_polyhedral_tool,TR:PACT2007:CommeEngineSoC,ReCoSoC:CoMAP,SAMOS4:PHRASE,PHRASE:toReconfig,ProSe:TR:2004,ProSe:TR:2003,EuroTeX2003:BibTeX++,WAPATV01-pips,sp97:pips,CPC96:pips:adding_dead_code_elimination,EPTM96:pips,pips:framework,RenPar'8:compilation_Phenix,techreport:sp96,linear_framework_HPF_CPC93}
  \putbib[biblio2e]
\end{bibunit}



\section{Hardware Secure Architectures for Distributed Computing ---
  \emph{S�curit� mat�rielle pour le calcul distribu�}}
\label{sec:secure-HPC}

\begin{bibunit}[liste_publications]
  \nocite{CESAR2008:archi_securisees,Lavoisier2008:systemes_repartis_en_action:8,TSI2008:CryptoPage,TR:TSI2008:CryptoPage,SGS2008:SAFESCALE,JoCV2008:CryptoPage,SympA:2008:CryptoPage_verif,SSTIC2007:CryptoPage,ACSAC-2006:CryptoPage,SympA:2006:CryptoPage/HIDE,ENSTB:2006:SAFESCALE,ENSTB:2006:secure_process_CryptoPage,TSI2005:CryptoPage,SympA2005:CryptoPage,ENSTB:2004:Linux_CryptoPage,rapport-TSI2004:CryptoPage,rapport:spec_OpenSmartCard,SympAAA2003-cryptopage,EUROSEC2001-cryptopage,SympA6-cryptopage,brevet-cryptopage-2000}
  \putbib[biblio2e]
\end{bibunit}


\section{High Performance Computing Architecture --- \emph{Architecture
    des ordinateurs � haute performance}}
\label{sec:arch-des-ordin}

\begin{bibunit}[liste_publications]
  \nocite{IFREMER2006:ordinateurs,EVD97,LaVillette:calculateurs,CEA_CUIC_parallelisme,synthese_POMP,activity_counter,techreport:activity_counter,MATISSES,these:keryell,controle_SIMD,POMP_PARLE,POMP_LIENS,POMP_luminy,POMP_PaM,POMP_Grenoble,magistere:keryell,DEA_Keryell}
  \putbib[biblio2e]
\end{bibunit}


\section{Multimedia, Networking \& P2P --- \emph{Multim�dia, r�seaux \&
    P�P}}
\label{sec:mult-et-rese}

\begin{bibunit}[liste_publications]
  \nocite{RenPar2003:ReActiVE,ENSTBr:ReActiVE}
  \putbib[biblio2e]
\end{bibunit}


\section{Low Performance Computing \smiley: Domotics ---
  \emph{Informatique � Faible Performance \smiley : domotique}}
\label{sec:inform-faible-perf}

Un clin d'\oe{}il � \textsc{lpf} (\emph{Low Performance Fortran}) \smiley.

A winckle to \textsc{lpf} (\emph{Low Performance Fortran}) $\smiley$.

\begin{bibunit}[liste_publications]
  \nocite{ICCHP2004:ametsa,ICOST2003:ametsa,SETIT2003:ametsa}
  \putbib[biblio2e]
\end{bibunit}


\section{Operating systems --- \emph{Syst�mes d'exploitation}}
\label{sec:syst-dexpl-oper}

\begin{bibunit}[liste_publications]
  \nocite{keryell01-JRES-amanda-etendu,keryell01-JRES-amanda,keryell01-JRES-cfengine-etendu,keryell01-JRES-cfengine}
  \putbib[biblio2e]
\end{bibunit}


\section{Teaching --- \emph{Enseignement}}
\label{sec:ense-teach}

Voir \emph{see} \LienPDF{\RacineHTML/cours}


\section{Presentations --- \emph{Pr�sentations}}
\label{sec:pres-pres}

See quite old stuff at \LienPDF{\RacineHTML/articles_et_programmes.html}


\section{D'autres articles et rapports --- \emph{Some other articles and reports}}
\label{sec:dautres-articles-et}

\begin{itemize}
\item Des rapports d'�tudiants -- \emph{Some student reports} :
  \LienPDF{\RacineHTML/eleves} ;
\item Au \href{http://departements.telecom-bretagne.eu/info}{Laboratoire d'Informatique} �
  l'\href{http://www.telecom-bretagne.eu}{T�L�COM Bretagne} -- \emph{At the
    \href{http://departements.telecom-bretagne.eu/info}{Computer Science
      Laboratory} of
    \href{http://www.telecom-bretagne.eu}{T�L�COM Bretagne}} :
    \LienPDF{http://departements.telecom-bretagne.eu/info/publications/}
\item Au \href{http://www.cri.ensmp.fr}{Centre de Recherche en
    Informatique} (CRI) des \href{http://www.ensmp.fr}{MINES ParisTech} -- \emph{At the
    \href{http://www.cri.ensmp.fr}{Computer Science Research Center} (CRI)
    of \href{http://www.ensmp.fr}{MINES ParisTech}} : \LienPDF{http://www.cri.ensmp.fr/classement/2013.html}
\item Au \href{http://www.di.ens.fr}{Laboratoire d'Informatique} (LIENS)
  de l'\href{http://www.ens.fr}{�cole Normale Sup�rieure} (ENS) --
  \emph{At the \href{http://www.di.ens.fr}{Computer Science Department}
    (LIENS) of the \href{http://www.ens.fr}{�cole normale sup�rieure}
    (ENS)} : mais o� est-ce bien pass� ???
\end{itemize}

\end{document}

%%% Local Variables: 
%%% mode: latex
%%% TeX-master: t
%%% End: 
